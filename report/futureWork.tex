The project as it is now, is not perfect. These are the improvements that can be done.
\subsection{Finding Clusters} 
The current cluster finding works pretty well. Bad (too short) clusters are
mostly cut out of the data before annotation gets done, and the way histograms
are used, make for a good method. There are, however, a couple of improvement
that could be implemented.

\begin{itemize}
 \item Improving training histograms

    The current training histograms are made with only one example. This means
    that if this example is not very representative, the resulting clusters are
    based on wrong examples. Better examples could be created by using more data
    than just the time window we selected for the loop, and then taking the
    mean.
 \item Comparing histograms

    The formula that is currently used for comparing histograms, is not perfect.
    There are other, better, ways to compare histograms, but in the time we had,
    we did not implement those.

 \item Using percentages
    
    Our current histogram compare method in \matlab, returns the best fitting
    histogram of the current time window. This can be 
    used for learning algorithms. A possible improvement would be to not return
    which histogram has the best fit, but to actually return the percentages of
    comparison between histograms. This way, for example, a decision tree
    learner could be able to distinguish floating behavior by saying the
    histogram percentage for floating should at least be 80\%. This will
    probably work better.
 
\end{itemize}
    
