%\section{Preprocessing}
The data we need to use is available from a database at the uva-bits website
\footnote{\url{https://public.flysafe.sara.nl/phppgadmin/} }. This database is often very slow 
and sometimes overloaded. This is why we downloaded all the data we required in one run. 
Before we can start in Matlab we preprocess this data to load only correct data in Matlab.

\subsection{Before Matlab}
\label{subsec:beforeMatlab}
Preprocessing of the data from a database dump is done in Java. Such data 
is checked for the following: 
\begin{enumerate}
    \item Is the data above the north sea? We are only interested in the birds behaviour 
    above the North Sea data from above the Wadden Sea or above Holland is removed.
    \item Is the data complete? If we require longitude and latitude every entry 
    without a value for these variables are excluded for the data for Matlab. 
    \item Is the data usable? Our method requires data to consist of long flights,
    not isolated data points. We call a sequence of useful data points a \textit{session}. 
    See subsection \ref{subsec:creatingSessions} for more details on sessions and how
    they are created.
\end{enumerate}

Furthermore the date time was rewritten to a time stamp. We wanted to time stamps to be 
as small as possible so instead of using standard Unix time we use \textit{seconds passed
since 
January 1 2008}. This could be done because the earliest data in the database was from after
this date. 

\subsection{Creating Sessions}
\label{subsec:creatingSessions}
As discussed in subsection \ref{subsec:beforeMatlab} a part of the script involves 
dividing the data points up in to sessions. Isolated data points should be excluded. 
When there is a too big of a time span between two data points, a new sessions should be 
created. Furthermore a session should not be too short and it should not contain too few
data points (the resolution of data points per minutes should be within a range of 
acceptable values). 

This has to be done because our method involves the detection of clusters and this
can only be done with an actual flight of the bird. 

We use the following parameters to define which sequences of data points can be called sessions: 
\begin{itemize}
    \item Total minutes described. This is currently set to \minimumSessionLengthMinutes minutes. Less than \minimumSessionLengthMinutes minutes of flight is not interesting.
    \item Range of resolution. We can filter for a range of data points per minutes. There is
no way to extract the same amount of information from low resolution data as from high 
resolution data. This is why we filter out the low resolution data. The resolution 
we now use is \resolutionRange data points per minute.
    \item The maximum amount of minutes between two data points. Sometimes the device
can't update at the preferred moments. We end a session at such a point because we need
the information of what the bird is doing \emph{all the time} to be able to cluster 
acceptably. 
\end{itemize}

The script outputs comma separated files with sessions of GPS data and, if available, comma
separated files with sessions of accelerometer data. Matlab can easily match these sessions
and so use at least the GPS data but also the accelerometer data if it is available. 

The amount of data that does not pass through the filters of this script depends on the 
parameters. The parameters decide when we call a sequence of data points a session. Is
it long enough? What should the resolution (data points per minute). How long can we receive
no data before we end a session? Etcetera ...

The parameters verified with our client result in 90\% of the data being thrown away. 
The motivation was: ``Better to use less data that you know is good than to use more data that 
is not as usable''. 

About 25\% of the GPS data also had accelerometer data.



