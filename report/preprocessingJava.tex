\subsection{Before Matlab: Sessions and Java}
The data we need to use is available from a database at the uva-bits website
\footnote{\url{https://public.flysafe.sara.nl/phppgadmin/} }. This database is often very slow 
and sometimes overloaded. This is why we downloaded all the data we required in one run. 

We made small script with Java to process this data. The input data (data directly from a 
database dump) was checked for the following: 
\begin{itemize}
    \item Is the data above the north sea? We are only interested in the birds behaviour 
    above the North Sea data from above the Wadden Sea or above Holland is removed.
    \item Is the data complete? If we require longitude and latitude every entry 
    without a value for these variables are excluded for the data for Matlab. 
    \item Is the data usable? Our method requires data to consist of long flights,
    not isolated data points. This method involved filtering out only those data points
    that together form a \textbf{session} that is long enough with a frequency that is
    decent enough. 
\end{itemize}

Furthermore the data time was rewritten to a timestamp. We wanted to time stamps to be 
as small as possible so instead of using standard Unix time we use \textit{seconds passed
since 
January 1 2008}. This could be done because the earliest data in the database was from after
this date. 

The script outputs comma separated files with sessions of GPS data and, if available, comma
separated files with sessions of accelerometer data. Matlab can easily match these sessions
and so use at least the GPS data but also the accelerometer data if it is available. 

The amount of data that does not pass through the filters of this script depends on the 
parameters. The parameters decide when we call a sequence of data points a session. Is
it long enough? What should the resolution (data points per minute). How long can we receive
no data before we end a session? Etcetera ...

The parameters verified with our client result in 90\% of the data being thrown away. 
The motivation was: ``Better to use less data that you know is good than to use more data that 
is not as usable''. 

About 25\% of the GPS data also had accelerometer data.



