After extracting the GPS and accelerometer data and molding it into a useable form, the importance of clustering became apparent. Looking at groups of points instead of individual points brought many advantages. They are easy to annotate, easy to classify and provide an intuitive way to look at the data. There were not many resources on clustering in this area, so this became our main focus. We tried multiple methods to cluster the data and found an approach using histograms most successful. By creating a histogram of both speeds and angles of the bird for a window of time, and pulling this window over the session, we check the resemblance of each point with example histograms for flying, diving and floating. These points are then grouped into clusters. Because of noise in the data and strange bird behaviour (not belonging to a category) there are still bad clusters. By restricting clusters to a certain size and resolution and tweaking the sample histograms we tried to minimize this. 

We then showed how such clusters can be used to quickly annotate data. The tool made for this report gives an overview of the clusters properties to help recognizing behaviour. We demonstrated the tool to a group of experts who were excited.

In the end we annotated a large part of the available data to use it for machine learning. The feature list we used was not very complete, also the annotation could have been beter. Using a decision tree algorithm we got an accuracy of $82\%$. This shows the method of clustering the data before annotating definitely has potential, and should be further explored.