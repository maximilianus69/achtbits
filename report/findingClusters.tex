 % Describes how we find the peaks/clusters in the data (Maarten de Waard will
 % fill this in)
 This section will discuss how we find clusters in sessions. A cluster is a 
part of a sessions in which the bird shows one type of behavior. Once we have
clusters we can annotate quickly and learn per cluster.

We approached this problem in two different manners. First we simply looked at
change in the acceleration. We spend a lot of time tweaking and optimizing the parameters
to find better clusters but in the end we had to conclude that this would never work. 
The second
approach works has it's origins in `computer vision' and works with histograms to 
match standard behaviour to parts of sessions.

\subsection{The First Approach: Finding Peaks}
 Probably the most important of the GPS data we get from \bits are the X and Y
 position and speed of the bird. There are two speed entries in the database we
 use. The first we use is `instantaneous speed'.    
This the speed the bird has on the time of
 connecting to the GPS, calculated from two subsequent measurements. The second
we use is 
 `trajectory speed'. This is speed calculated from two GPS entries. 
We decided to use the latter because it has a smaller error rate. 
 Our assumption is that we can cluster the data well 
 enough by only looking at speed differences. Figure \ref{fig:speed} shows
 why.
 % TODO: Dit figuurtje maken. Ik kom er namelijk net achter dat de derivative
 % misschien niet exact doet wat we willen, omdat het speed1 en speed2 apart van
 % elkaar gebruikt. Een goeie hiervoor is device_535, sessie_000.

\begin{figure}
  \centering
  \subfloat[Speed of the bird.]{\label{fig:speed1}\includegraphics[width=0.8\textwidth]{speed1}} \\
  \subfloat[Difference in speed.]{\label{fig:speed2}\includegraphics[width=0.8\textwidth]{speed2}} \\
  \caption{The difference is absolute because this makes finding peaks easier. The difference is sometimes a bit higher while the speed does not seem to differ at all. This means the bird changed directions.}
  \label{fig:speed}
\end{figure}

%\begin{figure}
%\centering
%\includegraphics[width=.8\textwidth]{speed.pdf}
%\caption{The speed of the bird (above) and the difference in speed (under). As
%you can see, the difference is absolute (because this makes finding peaks
%easier). The difference is some times a bit higher, while the speed does not
%seem to differ at all. This means the bird changed his direction.}
%\label{fig:speed}
%\end{figure}

As can be seen, the speed-difference is low on points where the bird is assumed
to be flying or sitting on the water. In other points the speed differs a lot.
On these places we can assume that the bird is foraging.


\subsubsection{Finding Peaks}
 Clustering, in our case, starts with finding peaks in these speed differences.
 The peaks indicate a change in the bird's behavior, or indicate that the bird
 is foraging. Finding these peaks is done in two steps:

 \begin{itemize}
    \item Check where the value of the speed difference is bigger than a certain
    threshold.
    \item Loop through these differences and place a marker before the threshold
    is crossed upwards, or after it has been crossed downwards. 
 \end{itemize}
 
 This creates a representation of a peak by marking its left and right side.
 This comes in handy, because we do not want these speed differences in our
 clusters as they would add noise to the learning data.

 \subsubsection{Grouping Peaks}
 For grouping peaks, we created another algorithm. This algorithm looks at three
 characteristics.  
 \begin{enumerate}
 \item The time elapsed between two peaks
 \item The difference between the current time elapsed between peaks, and the
 current cluster's average
 \item The difference between the time elapsed between the first peak of the
 cluster, and the average of the rest of the cluster.
 \end{enumerate}
 This way we can find the `chaos clusters', because the time between the peaks
 in these clusters is always between a certain threshold (currently set on
 \timeThreshold seconds). 
 When a peak is too far from the  current average, this almost always indicates
 a change in behavior, so a new cluster should be started.this is done by the
 second and third characteristic specified above.


It turned out however the data is too chaotic too group well in every situation. 
Also the difference between high and low resolution turned out to be troublesome. It
could not be done. We abandoned this train of thought and picked a more sophisticated 
method.


%%%%%%%%%%%%%%%%%%%%%%%%%%%%%%%%%%%%%%%%%%%%%%%%%%%%%%%%%%%%%%%%%%%%%%%%%%%%%%%%%%%
%%%%%%%%%%% AWESOME HISTOGRAM SUBSECTION BEGINT HIER
%%%%%%%%%%%%%%%%%%%%%%%%%%%%%%%%%%%%%%%%%%%%%%%%%%%%%%%%%%%%%%%%%%%%%%%%%%%%%%%%%%%


\subsection{The Second Approach: The Use of Histograms}
%The basis of the second approach relies on one important principle. 
%If a small part of a session is represented in such a manner that it can easily be 
%matched to a `model' (a part of a session of which we know what it is that is represented
%in the same manner) we can easily match the entire session to such a model. 
%
%The representation of a small session can easily be done with histograms. We count 
%the speed of the bird among standard intervals of the part of the session and do the
%same for the change of direction. The histogram we end up with can easily be matched
%to histograms of the models (floating, flying and diving). The similarity can now be 
%expressed in percentages: 100\% indicates a complete match and 0\% indicates a complete 
%mismatch. 
%
%Algorithm \ref{alg:hist} explains how we compare a session to the models
%
%Because a part of the session is matched we slide a window over the session and count 
%the speed and the angles of the directional changes in that window (\textit{getHistogram}) . 

This different version works on another basis: We create two histograms. One
contains the trajectory speeds of the bird and the other contains the
difference in the angle of the instantaneous speed. This means that we count
how many times the speed or angle is between certain values, and represent that
in an array with these counts. For further illustration see figure

The second method we tried, relies on a bit more data. Apart from the trajectory
speed, now also the instantaneous speed is used. The trajectory speed is relevant for
the speed of the bird and now we use the instantaneous speed for finding the angle
at the time of the GPS fix. 


% algoritme is fout: 
% en timeThreshold... 
% for i = windowsize/2 +1 to data.length-windowsize
\begin{algorithm}
\begin{codebox}
\Procname{$\proc{histogramCompare}(Data, timeThreshold)$}
\li $exampleHists \gets getExampleHists()$
\li \For $i \gets 1 + timeThreshold$ \To $\attrib{Data}{length}$ 
\li \Do
    $currentHists \gets getHistograms(i-timeThreshold, i+timeThreshold)$
    \li $histogramValues(i) \gets \Sigma \left| currentHists - exampleHists \right|$
\End
\li \Return histogramValues
\end{codebox}
\caption{Comparing example histograms to our data}
\label{alg:hist}
\end{algorithm}

We take an example of each kind of behavior (floating, flying and foraging).
This example can be compared with a bit of the data we are
clustering. For this we loop over the data with a time window. This is probably
better explained in the pseudocode in algorithm \ref{alg:hist}.

When using this method, some problems have to be solved.


\begin{figure}
  \centering
  \subfloat[Model for floating.]{\label{fig:clusteringRaw}\includegraphics[width=0.3\textwidth]{bar1speed}} 
  \subfloat[Model for flying.]{\label{fig:clusteringInterpolated}\includegraphics[width=0.3\textwidth]{bar2speed}} 
  \subfloat[Model for hunting.]{\label{fig:clusteringHists}\includegraphics[width=0.3\textwidth]{bar3speed}}
  \caption{Histograms that count the speed for the models of the three behaviors.}
  \label{fig:modelHistogramsSpeed}
\end{figure}


\begin{figure}
  \centering
  \subfloat[Model for flying.]{\label{fig:clusteringRaw}\includegraphics[width=0.3\textwidth]{bar2angle}} 
  \subfloat[Model for hunting.]{\label{fig:clusteringInterpolated}\includegraphics[width=0.3\textwidth]{bar3angle}} 
  \caption{Histograms that count the angle for the models of flying and hunting.}
  \label{fig:modelHistogramsAngle}
\end{figure}


%\begin{figure}
%\centering
%\includegraphics[width=.8\textwidth]{histogram.pdf}
%\caption{Example of a histogram. The x-axis contains the bins. These are the
%values in which the speeds could be. The height of the bars on the y-axis shows
%us how many times in a certain time span the bird had a certain speed.}
%\label{fig:histogram}
%\end{figure}


\subsubsection{Interpolating the Values}
The first problem we encounter is the difference in resolution of the data. This
would mean that when we are looking at high resolution data, and we compare it
with a low resolution example, line 4 of \ref{alg:hist} will not work. Therefore
we need to have exactly as many data points in the one, as in the other
histogram. 

This is easily achieved by interpolating the data. Since matlab has integrated
interpolation functions, we have chosen to interpolate with `Piecewise Cubic
Hermite Interpolating Polynomial (PCHIP)':, which is a bit more computationally
expensive than linear interpolation, but returns a smoother curve. This is
better, when using low resolution data, because it returns a bit smoother line.

\subsubsection{Comparing the Histograms}
As mentioned before, line 4 in algorithm \ref{alg:hist} compares two algorithms.
This returns a number close to zero, when the histograms are alike, and a high
number when they are different. Because this is not very intuitive, the program
recalculates this to percentages with equation \ref{eq:perc}.
\begin{equation}
\label{eq:perc}
percentage = 100 - \frac{Diff}{2 \sum Histogram} \times 100
\end{equation}

In this equation, Diff is the difference as calculated in algorithm
\ref{alg:hist} and the sum of a histogram is the count of all its entries. 

Now we have a percentage for each point, of how much it resembles each behavior.
A plot of this percentage is shown in the bottom of figure \ref{fig:clustering}.
This can be used to find clusters, because each time this differs, a new
cluster can be assumed to be starting.

\begin{figure}
  \centering
  \subfloat[Raw data of a session.]{\label{fig:clusteringRaw}\includegraphics[width=0.8\textwidth]{clustersHists1}} \\
  \subfloat[Interpolated data of a session.]{\label{fig:clusteringInterpolated}\includegraphics[width=0.8\textwidth]{clustersHists2}} \\
  \subfloat[Similarity of the flight to behavior percentages. Red indicates floating, green flying and blue hunting.]{\label{fig:clusteringHists}\includegraphics[width=0.8\textwidth]{clustersHists3}}
  \caption{Different plots of the clustering process. Red vertical bars indicate starts of clusters. Cyan vertical bars indicate ends of clusters.}
  \label{fig:clustering}
\end{figure}

%\begin{figure}
%\centering
%\includegraphics[width=\textwidth]{clustering.pdf}
%\caption{\small How clustering works. Above is the raw data. The marked line is the
%absolute speed of the bird. The red and blue lines under it are the speeds in
%the x and y direction.\\
%The middle image shows the interpolated data.\\
%The bottom image shows how much the point in the middle image resembles a
%certain kind of behavior. Red is floating, green is flying and blue is foraging.}
%\label{fig:clustering}
%\end{figure}

\subsubsection{Smoothing the current data}
After the previous step, the data sometimes has only one or two points above the
others, during a cluster. This could be due to a gps error or a sudden movement
of the bird. We do not want to see this as a new cluster, so the data should be
smoothed.

This is done in a simple and elegant way. A window of (in our case) 7 points is
moved over the data. The mode of these points should be the value of the middle
of the seven. This way, if less than four of the points have another value than
their surroundings, there are too little points to create a new cluster, and the
difference was probably noise. 
