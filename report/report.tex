\documentclass[a4paper,10pt]{article}
\usepackage[margin=2.5cm, nohead]{geometry}
\usepackage{palatino, url, multicol}
\usepackage{graphicx}
\usepackage{verbatim}
\usepackage[all]{xy}
\usepackage[dutch]{babel}

\newcommand{\bits}{BITS\xspace}

\title{Leren en Beslissen - ACHTBITS\\\large \textsc{Awesome Charadriiformes Toegepast
BIrd Tracking System}}
\author{Maarten de Waard\\5894883 \and Maarten Inja \\5872464 \and Jesse Eisses \\
6352189 \and Sosha Happel\\ 6273831}



%\email{mrtndwrd, maarten.inja, jesse.eisses, soshappel@gmail.com}
\begin{document}
\maketitle
\section{Introduction}
% I guess we will introduce the BITS, introduce some other stuff, maybe tell
% something about learning algorithms in general

\section{Preprocessing}
% Here we will discuss the preprocessing of the .csv files. This should consist
% of:
 \subsection{Getting the data we need}
 % Describing how we only get the data above the north sea. Maybe an appendix
 % should be added on how the java script can (should) be run.
 
 \subsection{Finding clusters in the data}
 % Describes how we find the peaks/clusters in the data (Maarten de Waard will
 % fill this in)
 This section will discuss how we find clusters in the raw data that has been
 given. This is needed, to annotate these clusters, so we can later on use a
 learning algorithm on specified parts of the data.

 Probably the most important of the GPS data we get from \bits are the X and Y
 position and speed of the bird. There are two speed entries in the database we
 use: One `instantaneous speed', the speed the bird has on the time of
 connecting to the GPS, calculated from two subsequent measurements, and a
 `trajectory speed', the speed the bird has between this, and the last
 measurement point. For clustering, we decided to use the latter, because this
 has a smaller error rate. Our assumption is that we can cluster the data good
 enough, by only looking at speed differences. Figure \ref{fig:speeds} shows
 why.
 % TODO: Dit figuurtje maken. Ik kom er namelijk net achter dat de derivative
 % misschien niet exact doet wat we willen, omdat het speed1 en speed2 apart van
 % elkaar gebruikt. Een goeie hiervoor is device_535, sessie_000.
 As can be seen, the speed-difference is low on points where the bird is assumed
 to be flying or sitting on the water. In other points the speed differs a lot.
 On these places we can assume that the bird is foraging.



 
 \subsection{Annotating the data}
 % Describes how we annotate (including how the tool works. Maybe an appendix
 % should be added on how the tool should be started and how it should be used.


\section{Learning}

\section{Conclusions and results}

\section{Improvements and future work}


\end{document}
