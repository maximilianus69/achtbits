\documentclass[a4paper, 11pt]{article}
\usepackage[T1]{fontenc}
\usepackage[margin=2.5cm, nohead]{geometry}
\usepackage{palatino, url, multicol}
\usepackage{amssymb, graphicx, fancyhdr, latexsym, url, verbatim}
\usepackage{algorithm, algorithmic}
\usepackage{hyperref}
\usepackage{clrscode3e}
\usepackage[all]{xy}
\usepackage[english]{babel}
\usepackage{matlabScripts}
\usepackage[font=small,format=plain,labelfont=bf,up,textfont=it,up]{caption}
%\usepackage{subfigure} % has to be loaded after caption to prevent clash Commented, because subfig is the newer (and presumably
% better) version of subfigure...
\usepackage{subfig}



\newcommand{\projectName}{ACHTBITS}
\newcommand{\projectAbbreviation}{Awesome
CHaradriiformes Toegepast BIrd Tracking System}
\newcommand{\bits}{BITS}

\addtolength{\footskip}{-90mm}
\addtolength{\headheight}{-05mm}
\addtolength{\headsep}{05mm}



\pagestyle{fancy}
\lhead{\projectName}
\rhead{\small \textsc{\projectAbbreviation}}
%\cfoot{\footnotesize \textit{ \projectAbbreviation}\\[0.1cm] \small \thepage}
%\cfoot{}
%\rfoot{\thepage}


\setlength{\parindent}{0pt}
\setlength{\parskip}{10pt}

\begin{document}

\begin{titlepage}

%\begin{minipage}{1in}
%\begin{tabular}{l}
%\includegraphics[width=0.5\textwidth]{./uom}
%\end{tabular}
%\end{minipage}
%\hfill
%\begin{minipage}{1in}
%\begin{tabular}{r}
%\includegraphics[width=0.5\textwidth]{./cse}
%\end{tabular}
%\end{minipage}


\begin{center}
% Upper part of the page

\textmd{Leren en Beslissen verslag}
\vfill
% Title
{ \huge \textbf{ACHTBITS} \\\large \textsc{Awesome Charadriiformes Toegepast
BIrd Tracking System}
}\\[0.4cm]
%\begin{center}
\vfill
By\\
\vfill
%\large\textbf{{Maarten de Waard \\ 5894883},  {Maarten Inja \\ 5872464},  {Jesse Eisses \\
%6352189}, {Sosha Happel \\ 6273831}}
\begin{tabular}{cccc}
Jesse Eisses & Sosha Happel & Maarten Inja & Maarten de Waard \\ 
6352189 & 6273831 & 5872464 & 5894883 
\end{tabular}
\vfill
\{mrtndwrd, maarten.inja, jesse.eisses, soshappel\}@gmail.com
%6352189 \and Sosha Happel\\ 6273831}
\end{center}

%\begin{center}
%\textmd{MSc 2008 Registration No: 088274C}\\
%\vfill
%\large\textbf{Supervisor: Dr. Buddhinath Jayatilleke}
%\vfill
%May 2009\\
%\end{center}

%\vfill
%\begin{center}
%This proposal is submitted in partial fulfillment of the requirement of\\
%Master of Science in Computer Science\\
%in the\\
%Department of Computer Science and Engineering\\
%Faculty of Engineering\\
%University of Moratuwa\\
%\end{center}
%\vfill


%\end{center}

\end{titlepage}


%\thispagestyle{empty}
\vspace*{00mm}
\tableofcontents
\newpage


%%%%%%%%%%%%%%%% variables

\newcommand{\minimumSessionLengthMinutes}{60}
\newcommand{\resolutionRange}{$0.2$ - $\infty$ }
\newcommand{\sessionSeparatorLengthMinutes}{15}
%% The current value of the timeThreshold given to awesomizeClusters.
\newcommand{\timeThreshold}{1200}

% sja, ik tiep hier maar wat hoor
\begin{abstract}
It is hard to classify the behavior of the sea gull above the North Sea. 
There is GPS data available and sometimes even accelerometer data but it 
remains troublesome to annotate data point by point. In this report we 
describe how we cluster data using histograms. We also describe  the annotation
tool we built with which we can annotate clustered data points more easily. 
\end{abstract}

\section{Introduction}
UvA Bird Tracking System\footnote{\url{http://www.uva-bits.nl/} } is a research 
project that studies the behavior of
birds. 
In contrast to some other birds that are being tracked the Sea Gull flies above the 
sea where it is a lot harder to watch and see what it is doing. We have data available
collected by a small device (see figure \ref{fig:gpsDevice}) that can be attached to 
the back of a bird. 

\begin{figure}
    \center
    \includegraphics[width=7cm]{device}
    \caption{A small GPS device that can also transmit accelerometer data and some other
information}
    \label{fig:gpsDevice}
\end{figure}

The data, which consists of thousands of data points, is hard to read and to annotate. 
There are tools that can rewrite the data to a format that can be read by Google Earth. 
Viewing this data in Google Earth is a bit helpful to get a general idea of what such 
a gull might be doing but it does not allow for easy manipulation of the data. 

During four weeks we worked to preprocess, cluster, annotate
and lastly: classify the data. 

% perhaps more about the (other) research goals




\section{Preprocessing}


\subsection{Before Matlab: Sessions and Java}
The data we need to use is available from a database at the uva-bits website
\footnote{\url{https://public.flysafe.sara.nl/phppgadmin/} }. This database is often very slow 
and sometimes overloaded. This is why we downloaded all the data we required in one run. 

We made small script with Java to process this data. The input data (data directly from a 
database dump) was checked for the following: 
\begin{itemize}
    \item Is the data above the north sea? We are only interested in the birds behaviour 
    above the North Sea data from above the Wadden Sea or above Holland is removed.
    \item Is the data complete? If we require longitude and latitude every entry 
    without a value for these variables are excluded for the data for Matlab. 
    \item Is the data usable? Our method requires data to consist of long flights,
    not isolated data points. This method involved filtering out only those data points
    that together form a \textbf{session} that is long enough with a frequency that is
    decent enough. 
\end{itemize}

Furthermore the data time was rewritten to a timestamp. We wanted to time stamps to be 
as small as possible so instead of using standard Unix time we use \textit{seconds passed
since 
January 1 2008}. This could be done because the earliest data in the database was from after
this date. 

The script outputs comma separated files with sessions of GPS data and, if available, comma
separated files with sessions of accelerometer data. Matlab can easily match these sessions
and so use at least the GPS data but also the accelerometer data if it is available. 

The amount of data that does not pass through the filters of this script depends on the 
parameters. The parameters decide when we call a sequence of data points a session. Is
it long enough? What should the resolution (data points per minute). How long can we receive
no data before we end a session? Etcetera ...

The parameters verified with our client result in 90\% of the data being thrown away. 
The motivation was: ``Better to use less data that you know is good than to use more data that 
is not as usable''. 

About 25\% of the GPS data also had accelerometer data.




% Here we will discuss the preprocessing of the .csv files. This should consist
% of:
 % Describing how we only get the data above the north sea. Maybe an appendix
 % should be added on how the java script can (should) be run.
 
\section{Finding Clusters in the Data}
 % % Describes how we find the peaks/clusters in the data (Maarten de Waard will
 % fill this in)
 This section will discuss how we find clusters in sessions. A cluster is a 
part of a sessions in which the bird shows one type of behavior. Once we have
clusters we can annotate quickly and learn per cluster.

We approached this problem in two different manners. First we simply looked at
change in the acceleration. We spent a lot of time tweaking and optimizing the parameters
to find better clusters but in the end we had to conclude that this would never work. 
The second
approach has it's origins in `computer vision' % Is dit waar!?
and works with histograms to 
match standard behaviour to parts of sessions.

\subsection{The First Approach: Finding Peaks in Acceleration}
 Probably the most important of the GPS data we get from \bits are the X and Y
 position and speed of the bird. There are two speed entries in the database we
 use. The first we use is `instantaneous speed'.    
This is the speed the bird has on the time of
 connecting to the GPS, calculated from two subsequent measurements. The second
we use is 
 `trajectory speed'. This is speed calculated from two GPS locations. 
We decided to use the latter because it has a smaller error rate. 
 Our assumption was that we could cluster the data well 
 enough by only looking at speed differences. 
 % TODO: Dit figuurtje maken. Ik kom er namelijk net achter dat de derivative
 % misschien niet exact doet wat we willen, omdat het speed1 en speed2 apart van
 % elkaar gebruikt. Een goeie hiervoor is device_535, sessie_000.

\begin{figure}
  \centering
  \subfloat[Speed of the bird.]{\label{fig:speed1}\includegraphics[width=0.8\textwidth]{speed1.png}} \\
  \subfloat[Difference in speed.]{\label{fig:speed2}\includegraphics[width=0.8\textwidth]{speed2}} \\
  \caption{The difference is absolute because this makes finding peaks easier. The difference is sometimes a bit higher while the speed does not seem to differ at all. This means the bird changed directions.}
  \label{fig:speed}
\end{figure}

%\begin{figure}
%\centering
%\includegraphics[width=.8\textwidth]{speed.pdf}
%\caption{The speed of the bird (above) and the difference in speed (under). As
%you can see, the difference is absolute (because this makes finding peaks
%easier). The difference is some times a bit higher, while the speed does not
%seem to differ at all. This means the bird changed his direction.}
%\label{fig:speed}
%\end{figure}

As can be seen in figure \ref{fig:speed}, the speed-difference is low on points where the bird is assumed
to be flying or sitting on the water, but is high for a short period when the bird transistions from one to another. In other points the speed differs a lot over a longer period of time. On these intervals we can assume that the bird is foraging.


\subsubsection{Finding the Peaks}
 Clustering, in our case, starts with finding peaks in these speed differences.
 The peaks indicate a change in the bird's behavior, or indicate that the bird
 is foraging. Finding these peaks is done in two steps:

 \begin{itemize}
    \item Check where the value of the speed difference is bigger than a certain
    threshold.
    \item Loop through these differences and place a marker before the threshold
    is crossed upwards, or after it has been crossed downwards. 
 \end{itemize}
 
 This creates a representation of a peak by marking its left and right side.
 This has the advantage that we do not get these speed differences in our
 clusters as they would add noise to the learning data.

 \subsubsection{Grouping the Peaks}
 For grouping peaks, we created another algorithm. This algorithm looks at three
 characteristics.  
 \begin{itemize}
 \item The time elapsed between two peaks
 \item The difference between the current time elapsed between peaks, and the
 current cluster's average
 \item The difference between the time elapsed between the first peak of the
 cluster, and the average of the rest of the cluster.
 \end{itemize}
 This way we can find the `chaos clusters', because the time between the peaks
 in these clusters is always between a certain threshold (currently set on
 \timeThreshold seconds). 
 When a peak is too far from the  current average, this almost always indicates
 a change in behavior, so a new cluster should be started.this is done by the
 second and third characteristic specified above.

It turned out however the data is too chaotic too group well in every situation. 
Also the difference between high and low resolution turned out to be troublesome. It
could not be done. We abandoned this train of thought and picked a more sophisticated 
method.


%%%%%%%%%%%%%%%%%%%%%%%%%%%%%%%%%%%%%%%%%%%%%%%%%%%%%%%%%%%%%%%%%%%%%%%%%%%%%%%%%%%
%%%%%%%%%%% AWESOME HISTOGRAM SUBSECTION BEGINT HIER
%%%%%%%%%%%%%%%%%%%%%%%%%%%%%%%%%%%%%%%%%%%%%%%%%%%%%%%%%%%%%%%%%%%%%%%%%%%%%%%%%%%


\subsection{The Second Approach: The Use of Histograms}
\begin{figure}
  \centering
  \subfloat[Model for floating.]{\label{fig:clusteringRaw}\includegraphics[width=0.3\textwidth,height=1in]{bar1speed}} 
  \subfloat[Model for flying.]{\label{fig:clusteringInterpolated}\includegraphics[width=0.3\textwidth,height=1in]{bar2speed}} 
  \subfloat[Model for hunting.]{\label{fig:clusteringHists}\includegraphics[width=0.3\textwidth,height=1in]{bar3speed}}
  \caption{Histograms that count the speed for the models of the three
  behaviors, using the following bins: [0, 1, 2, 4, 5, 10, 20, 40, 60, 120]}
  \label{fig:modelHistogramsSpeed}
\end{figure}
\begin{figure}
  \centering
  \subfloat[Model for
  flying.]{\label{fig:clusteringRaw}\includegraphics[width=0.3\textwidth,height=1in]{bar2angle}} 
  \subfloat[Model for hunting.]{\label{fig:clusteringInterpolated}\includegraphics[width=0.3\textwidth, height=1in]{bar3angle}} 
  \caption{Histograms that count the angle for the models of flying and hunting,
  using the following bins: [0, 5, 10, 30, 60, 180, 360]. There is no model for
  floating, because  the direction of floating could not be measured properly
  using instantaneous speed.}
  \label{fig:modelHistogramsAngle}
\end{figure}
This different version works on another basis: We create two histograms. One
contains the trajectory speeds of the bird and the other contains the
difference in the angle of the instantaneous speeds. This means that we count
how many times the speed or angle is between certain values, and represent that
in an array with these counts. For further illustration see figure
\ref{fig:modelHistogramsSpeed} and \ref{fig:modelHistogramsAngle}.


We take an example of each kind of behavior (floating, flying and foraging).
This example can be compared with a bit of the data we are
clustering. For this we loop over the session data, each time selecting
\windowSize.

This method consists of five steps:
\begin{itemize}
    \item Interpolating the values
    \item Creating the histograms
    \item Comparing the histograms
    \item Smoothing the histogram values
    \item Finding the clusters
\end{itemize}

Using these steps, the program can loop over the session data, using a certain time
window. Histograms made from the time window can then be compared to the
histograms of the examples (which equally sized) and a
percentage of how different they are can be calculated. The session can now be clustered by
adding cluster beginnings and endings where the values of the histogram
differences (so the values in figure \ref{fig:clusteringHists}) cross each
other. The histogram that has the highest values, is also put in our feature
vector for learning purposes.

\begin{figure}
  \centering
  \subfloat[Raw data of a session.]{\label{fig:clusteringRaw}\includegraphics[width=0.8\textwidth]{clustersHists1}} \\
  \subfloat[Interpolated data of a session.]{\label{fig:clusteringInterpolated}\includegraphics[width=0.8\textwidth]{clustersHists2}} \\
  \subfloat[Similarity of the flight to behavior percentages. Red indicates floating, green flying and blue hunting.]{\label{fig:clusteringHists}\includegraphics[width=0.8\textwidth]{clustersHists3}}
  \caption{Different plots of the clustering process. Red vertical bars indicate
  beginnings of clusters. Cyan vertical bars indicate ends of clusters.}
  \label{fig:clustering}
\end{figure}

\subsubsection{Interpolating the Values}
The data consists of sessions with different resolutions. Even during the
session, the resolution can differ. This
would mean that when we are looking at high resolution data, and we compare it
with a low resolution example, calculating an accurate difference value between
two histograms would get very difficult. Therefore
we need to have exactly as many data points in the example, as in the current 
histogram. 

This is easily achieved by interpolating the data. Since \matlab has integrated
interpolation functions, we have chosen to interpolate with `Piecewise Cubic
Hermite Interpolating Polynomial (PCHIP)', which is a bit more computationally
expensive than linear interpolation, but returns a smoother curve. 

\subsubsection{Creating the Histograms}
%How we cretae heistrioangsams:
In \matlab, creating histograms is not a problem. The function \texttt{histc} is
used, and returns a division of the data, over the specified bins. A histogram
in \matlab is formatted like a 1 by N matrix. These
histograms can be compared as described in the next part.

\subsubsection{Comparing the Histograms}
Equation \ref{eq:diff} compares two algorithms, and returns a value closer to
zero when the histograms are the same, going up when they differ more.
\begin{equation}
\label{eq:diff}
Diff = \sum ( \left| exampleHistogram - currentHistogram \right| ) \vspace{10pt}
\end{equation}
 
Because the maximum value of $Diff$ depends on the time window, the program converts
it to percentages. That is done with equation \ref{eq:perc}, where the sum of a
histogram is the same as the number of entries it has.

\begin{equation}
\label{eq:perc}
percentage = 100 - \frac{Diff}{2 \sum Histogram} \times 100
\end{equation}

Now we have a percentage for each point, of how much it resembles each behavior.
A plot of this percentage is shown in figure
\ref{fig:clusteringHists}.
This can be used to find clusters, because each time this differs, a new
cluster can be assumed to be starting.

\subsubsection{Smoothing the Histogram Values}
After the previous step, the data sometimes has only one or two points above the
others, during a cluster. This could be due to a gps error or a sudden movement
of the bird. We do not want to see this as a new cluster, so the data should be
smoothed.

This is done in a simple and elegant way. A window of (in our case) 7 points is
moved over the data. The mode of these points should be the value of the middle
of the seven. This way, if less than four of the points have another value than
their surroundings, there are too little points to create a new cluster, and the
difference was probably noise. 

\subsubsection{Finding the Clusters}
In this smoothed data, the clusters can easily be found. The program saves the
timestamp of each time the histogram values differ, this way returning a matrix
with two columns: Begin time, and end time. 

The clusters shorter than 15 minutes, are deleted. Allthough
this has as a result that we do not use all the data for learning, this does
select the best usable data, and uses that. the clusters of approximately 5 to
15 minutes would be worthless learning data, because they are too short to
depict actual behavior.



 \subsection{Annotating the data}
 % Describes how we annotate (including how the tool works. Maybe an appendix
 % should be added on how the tool should be started and how it should be used.









\section{Learning}



The focus of this report was cluster finding by unsupervised learning, which gave some workable results. The found clusters can be used to identify behaviour using machine learning techniques. We will give an example of how this can be done in a way that is easy to extend. We will use the clusters that we manually labeled and attach a set of features to them (a set of properties that describe it). With the cluster labels and their features we train a learning algorithm and test its accuracy.

 \subsection{Features}
 The features we came up with consist of numeric values and can be derrived from the cluster data. Some are used for manual annotation and some are values that we hope the learning algorithm can use to find hidden relations. Here is a list with their descriptions:

 \begin{description}
  \item[Duration] The duration of the cluster in seconds.
  \item[Average speed] Average speed during the cluster in km/h.
  \item[Height difference] An estimation of the height traveled in meters. The height is very unreliable, even after removing outliers, so we can not just use the sum of the values. Instead we calculate the derivative of the height numerical (the verticale speed) and take the average value. To make this value more intuitive we multiply it by the duration of the cluster, which results in the average vertical distance traveled.
  \item[Ground distance] Distance between cluster begin- and endpoint in km. 
  \item[Total distance] Total distance traveled in km.
  \item[Angle variance] The average change of direction in degrees per minute. On low resolution it can look like the bird is flying in a straight line, while he is really making sharp turns. Figure \ref{fig:anglevar1} shows a bird\footnote{bird 344, session 10} who seems to be flying straight on his trajectory, while his speeds indicate he is not. For this reason we do not use the angle between two data points, but the angle between two \emph{instantaneous speed vectors}. These are 3d vectors indicating the motion of the bird measured on a small time interval.

\begin{figure}[htb!]
\begin{center}
\includegraphics[scale=0.8]{anglevar2.png}
\end{center}
\caption{Trajectory of a bird with his instantaneous speeds} 
\label{fig:anglevar1}
\end{figure}

  There are some problems with this feature, because the instantaneous speed is unreliable. When the resolution is high the time passed between two points is very low, but the angle between them can be a bit jumpy. This results in large changes in angle while the bird is flying straight. We try to limit this error by sampling the data when the resolution is too high (above 1 dat/min is the limit used in our code).
  \item[Distance difference] The difference between \emph{ground distance} and \emph{total distance}. When the bird is hunting he often flies in circles and returns near his startpoint, which is easy to read in this variable (we include this because the learning algorithm can not see spatial information in the trajectory).
  \item[Resolution] The average amount of data points per minute in the clusters data. Not very useful for us but maybe for the learning algorithm.
  \item[Fourier frequencies] Maximum frequency in the fourier transform in the $x$, $y$ and $z$ directions (three features). The fourier transform is an operation that expresses a function of time as a function of frequencies. This is a logical way to express acceleremeter data. For the implementation we used a method called the \emph{Fast Fourier transform}\footnote{TODO Reference to article in repo}. 

  \begin{figure}[htb!]
    \centering
    \subfloat[Accelerometer data, on the x-axis time (ms)]{
      \includegraphics{acc1}
    }\\
    \subfloat[Fourier transform of the above data, on the x-axis frequency (Hz)]{
      \includegraphics{fourier2}
    }\\
    \caption{The fourier transform on a fragment of accelerometer data}
    \label{fig:fourier}
  \end{figure}

  Figure \ref{fig:fourier} shows an example of a fourier transform on accelerometer data. There is a peak at 3 Hz, which means the accelerometer data has fragments recurring at that frequency. The peak at 0 Hz is an aspect of the Fourier transform and can be ignored. 

  Our fourier transform only returns the maxiumum frequencies of one point of accelerometer data. We currently use the center point for this feature, which is not te best method because that point could contain noise. This could be improved by taking the average of the frequencies of all accelerometer data in the cluster.
  \item[Previous cluster] The id of the previous cluster. This may be needed to recognize some behaviour, like digesting (after hunting).
 \end{description}

 We store the features in a Matlab array and save them to a csv file.

\subsection{Adding Features}
The GPS and accelerometer data can be used to create different features not mentioned above. We did not have the resources to do research in finding the best features, but we wanted to be able to add new ones without re-annotating the data. All features are calculated in \verb|createClusterFeatures.m| and are then saved along with their annotation. When a new feature is thought up it can be appended to the return value of that function. Then \verb|raloadFeatures.m| can be used to recreate the features, transferring all previous annotations. 

 \subsection{Results}
 We then appended all the csv files into a large database of training data. We removed \emph{unknown} and \emph{bad} labeled clusters as they indicate bad clustering or insufficient knowledge of birds. To improve the result we also combined \emph{digesting} and \emph{sleeping} clusters into one class. They can be seperated in future experiments, but with our current features and annotations an algorithm will not find a difference. The training data can then be loaded into a machine learning program. We chose for WEKA\footnote{http://www.cs.waikato.ac.nz/ml/weka/} because it is easy to use and it supports many learning algorithms.

 For the tests we use following dataset:
\begin{itemize}
\item 1637 annotated clusters
\item 12 features
\item 3 classes (Flying, Floating and Diving)
\end{itemize}


WEKA gave the following results, using 10-fold crossvalidation for testing:

\begin{center}
\begin{tabular}{r|c}
	\textnormal{Method} & \textnormal{Precision} \\ \hline 
	Tree J48 & 81\% \\
	Logistic & 82\% \\
  Perceptron & 82\% \\
\end{tabular}
\end{center}


The learning methods yield very similar results. Figure \ref{fig:tree} shows the J48 tree that was generated. 

\begin{figure}
\begin{center}
  \xymatrix{
    && *+<12pt>[F-:<12pt>]{\txt{avgSpeed}}\ar[dl]|{<=12.315}\ar[dr]|{>12.315}  & & & &\\
    &*+<12pt>[F=:<12pt>]\txt{floating} & & *+<12pt>[F-:<12pt>]\txt{grndDist}\ar[dl]|{<=2.6718}\ar[dr]|{>2.6718} & &\\
    && *+<12pt>[F-:<12pt>]\txt{distDiff}\ar[dl]|{<=0.10189}\ar[dr]|{>0.10189} & & *+<12pt>[F=:<12pt>]\txt{flying} &\\
    &*+<12pt>[F=:<12pt>]\txt{flying} & & *+<12pt>[F-:<12pt>]\txt{grndDist}\ar[dl]|{<=1.1718}\ar[dr]|{>1.1718} & &\\
    && *+<12pt>[F=:<12pt>]\txt{diving} & & *+<12pt>[F-:<12pt>]\txt{angleVar}\ar[dl]|{<=0.1878}\ar[dr]|{>0.1878}\\
    && & *+<12pt>[F=:<12pt>]\txt{flying}& & *+<12pt>[F-:<12pt>]\txt{distDiff}\ar[dl]|{<=0.6530}\ar[dr]|{>0.6530} & \\
    && & & *+<12pt>[F=:<12pt>]\txt{flying}& & *+<12pt>[F=:<12pt>]\txt{diving}
  }
  \caption{Visualisation of the decision tree generated by WEKA}
  \label{fig:tree}
\end{center}
\end{figure}

Most of the errors were made in classifying \emph{diving}, the cunfusion matrix shows this:
\begin{verbatim}
=== Confusion Matrix ===

   a   b   c   <-- classified as
 519  26  22 |   a = floating
  35 632  77 |   b = flying
  64  99 163 |   c = diving
\end{verbatim}
The following table shows the prediction accuracy for each class:
\begin{center}
\begin{tabular}{l|l}
	\textnormal{Class} & \textnormal{Correctly classified} \\ \hline 
	Floating & 91.5\% \\
	Flying &  85.1\% \\
	Diving & 50.0\% \\
\end{tabular}
\end{center}

The data was annotated by 4 different people with very little knowledge of birds, which is a bit reason for the misclassifications. 


\section{Conclusions and results}

\section{Improvements and future work}


\end{document}
