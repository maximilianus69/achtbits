UvA Bird Tracking System\footnote{\url{http://www.uva-bits.nl/} } is a research 
project that studies the behavior of
birds. 
In contrast to some other birds that are being tracked the Sea Gull flies above the 
sea where it is a lot harder to watch and see what it is doing. We have data available
collected by a small device (see figure \ref{fig:gpsDevice}) that can be attached to 
the back of a bird. 

\begin{figure}[h]
    \center
    \includegraphics[width=7cm]{gpsDevice}
    \caption{A small GPS device that can also transmit accelerometer data and some other
information}
    \label{fig:gpsDevice}
\end{figure}

The data, which consists of thousands of data points, is hard to read and to annotate. 
There are tools that can rewrite the data to a format that can be read by Google Earth. 
Viewing this data in Google Earth is a bit helpful to get a general idea of what such 
a gull might be doing but it does not allow for easy manipulation of the data. 

During four weeks we worked to preprocess the data, cluster the data, annotate the 
data and lastly: classify the data. 

