\subsubsection{Downloading Data From the Database}
The data we need to use is available from a database at the UvA-bits website
\footnote{\url{https://public.flysafe.sara.nl/phppgadmin/} }. This database is often very slow 
and sometimes overloaded. This is why we downloaded all the data we required in one run. 
Before we can start in \textsc{matlab} we preprocess this data to load only correct data
in it.

The query we use to download the GPS data is of the following format: 

%\lstinputlisting[language=SQL]{queryies.sql}
\begin{verbatim}
select *
from gps.uva_tracking_speed_3d_limited
where device_info_serial = 344 
and date_time > '2008-07-01 00:00:00' 
and date_time < '2012-08-01 00:00:00';

\end{verbatim}

The query we use to download the accelerometer data is of the following format:
\begin{verbatim}
select a.device_info_serial, a.date_time, a.index,
(a.x_acceleration-d.x_o)/d.x_s as x_cal,
(a.y_acceleration-d.y_o)/d.y_s as y_cal,
(a.z_acceleration-d.z_o)/d.z_s as z_cal
from gps.uva_device_limited d
full outer join gps.uva_acceleration_limited a
using (device_info_serial)
where a.device_info_serial = 344 and
a.device_info_serial = d.device_info_serial and
a.date_time > '2008-07-01 00:00:00' and a.date_time < '2012-08-01 00:00:00'
order by a.date_time, a.index;
\end{verbatim}

Only the \textit{device\_id} should change to select different. Obviously more complex queries could 
decrease the magnitude of the task of our preprocessing script but we wanted to keep the
work for the database to a minimum for (reasons discussed earlier) and this seemed the
easiest way to do this. 

The result of the queries should be dumped in one folder to `comma separated files' (csv) files.
The data filenames should be the following : DATA\_ALL\_[GPS/ACCELEROMETER]\_BIRD\_X.csv for 
GPS or accelerometer data for \textit{device\_id} X. 

\subsubsection{Creating Session Files}
Now the data is downloaded it can be preprocessed before \textsc{matlab} can do its magic. 
This can done for a single \textit{device\_id} by \textit{PreprocessCsvFiles.java} or 
for multiple \textit{device\_id}s by \textit{Wrapper.java}. The Wrapper simply uses the PreprocessCsvFiles. 

A special \textit{config.txt} that can be adjusted for those who do not like to program 
Java can be used to customize the script PreprocessCsvFiles. This configuration file
has a couple of lines with description=value. The value (the part after an equal sign)
can be adjusted to whatever legal
value one
would like. The description (the part before an equal sign) can be anything one would
except more equal signs. Do not change the order of the lines.

The only variables that are not self explanatory are those that decide which columns
of the input data file should be used (those columns that should be extracted from the
input files and should be put in the output files). 
This is an array with integers and there are 
two for these, one for the GPS data files and one for the accelerometer data files. 
Luckily, these should not have to be changed since the current values extract all the 
relevant
columns. 

Both scripts work with command line arguments. For more elaboration one what the
scripts exactly do one could check our report for an extensive list or the README file for 
a short overview. 

Command line arguments for PreprocessCsvFiles (do not change the order of these arguments): 
\begin{itemize}
    \item The .csv file with GPS data.
    \item The directory in which to create a folder with the session files.
    \item (Optional argument) The .csv file with accelerometer data.
\end{itemize}

Command line arguments for Wrapper (do not change the order of these arguments): 
\begin{itemize}
    \item The output directory in which to create folders with the sessions.
    \item The input director which contains all the data files (of the known format). 
\end{itemize}

















