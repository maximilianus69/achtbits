\documentclass{beamer}

\usepackage{amsmath}
\usepackage{graphicx}
\usepackage{enumerate}
\usepackage{subfigure}

\graphicspath{{images/}}

\title{Acht-Bits \\ \small }
\author{Sosha Maarten Maarten Jesse}
\institute{UvA}

\begin{document}
\maketitle

\begin{frame}{Database \& Tools}

\begin{block}{GPS Data}
The transmitter collects accelerometer and GPS data. The accelerometer data is very big so we decided to start with only the GPS data.
\begin{itemize}
\item Get latitude, longitude and altitude data
\item Only use points above the North Sea
\end{itemize}
\end{block}

\begin{block}{Software}
We use the following tools:
\begin{itemize}
\item Matlab for calculations
\item Google Earth for visualizing bird flight in 3D (?)
\end{itemize}
\end{block}

\end{frame}

\begin{frame}{Clusters}
\begin{block}{Clustering}
We want our program to detect intervals in which the bird is performing a behaviour; creating clusters of data that are easy to label. 
To detect clusters we search for \emph{change} in speed and direction, the second order derivative of the location. The plan:
\begin{itemize}
\item Calculate the derivative of the speed as a vector
\item If this vector is above a threshold it is a cluster transition
\item Areas with unstable vectors also become clusters
\item Manually label the clusters
\end{itemize}
\end{block}

\begin{block}{Learning}
We then attach features (representations of the GPS data) to each cluster so we can use machine learning to automatically label data.
\end{block}

\end{frame}

\end{document}