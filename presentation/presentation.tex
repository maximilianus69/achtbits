\documentclass{beamer}
\usepackage[utf8x]{inputenc}
%\usepackage{default}
\usepackage{verbatim}


\newcommand{\projectName}{ACHTBITS}
\newcommand{\projectAbbreviation}{Awesome
CHaradriiformes Toegepast BIrd Tracking System}


\title{\projectName}
\subtitle{\projectAbbreviation}
\author{Jesse Eisses, Sosha Happel, Maarten Inja and Maarten de Waard}
\institute{UvA}
\usetheme{Berkeley}
\newcommand{\slide}[2]
{
\begin{frame}
\frametitle{#1} 


#2

\end{frame}
}



\begin{document}
\begin{frame}
\titlepage
\end{frame}



\section{Introduction}
\slide{Het Project}
{
UvA-bits verzamelt GPS en accelerometerdata van vogels.\\
Onderzoeksdoelen:
\begin{itemize}
	\item Automatische segmentatie van datapunten
	\item Classificatie van gedrag boven de Noordzee
\end{itemize} 
}

\slide{Problemen}
{
\begin{itemize}
	\item Veel onbruikbare data
	\item Weinig features
	\item Geen geannoteerde data
	\item Annoteren is erg lastig
\end{itemize} 
}

\slide{Plan}
{
Zo veel mogelijk automatisch
\begin{itemize}
	\item Data opsplitsen in bruikbare sessies
	\item Sessies automatisch segmenteren in clusters
	\item Nieuwe features maken voor clusters
	\item Annoteerproces vergemakkelijken
	\item Clusters classificeren met nieuwe features
\end{itemize} 
}

\section{Preprocessing}
\slide{Data opsplitsen in sessies}
{
\begin{itemize}
    \item Alles boven de noordzee weggooien
    \item Data mag niet `NULL' zijn.
    \item Minimaal 1 uur
    \item Gewenste resolutie
    \item Geen te grote gaten tussen datapunten
\end{itemize}
}

\slide{Finding clusters (old method)}
{
     Alleen kijken naar snelheid.\\
     Pieken vinden:
 \begin{itemize}
    \item Kijk of de waarde van het snelheidsverschil groter is dan een
    threshold
    \item Loop door deze waardes, en markeer een cluster voordat de waarde omhoog
    gaat, of nadat deze omlaag gaat.
 \end{itemize}
    Pieken clusteren:
    \begin{itemize}
 \item The time elapsed between two peaks
 \item The difference between the current time elapsed between peaks, and the
 current cluster's average
 \item The difference between the time elapsed between the first peak of the
 cluster, and the average of the rest of the cluster.
 \end{itemize}

}

\slide{Finding clusters (new method)}
{
    
}



\section{Results}
\slide{Annotation tool}
{
	
}

\slide{Classification}
{
	
}

\section{Conclusion}

\slide{Conlusion}
{
	
}

\slide{Future work}
{
	
}
\end{document}
